\documentclass[12pt]{article} % Define la clase del documento, en este caso, un artículo

% Paquetes de idioma y codificación
\usepackage[utf8]{inputenc}
\usepackage[T1]{fontenc}
\usepackage[spanish]{babel}  % Ajusta el idioma del documento a español.

% Paquete de geometría para configurar márgenes y tamaño de papel
\usepackage[letterpaper, margin=3cm]{geometry}

% Paquetes de tipografía
\usepackage{mathptmx}    % Usa Times New Roman como fuente.
\usepackage{microtype}   % Mejora la justificación del texto.

% Paquetes para manejo de colores y gráficos
\usepackage{xcolor}      % Define y utiliza colores.
\usepackage{graphicx}    % Permite la inserción de imágenes.
\usepackage{tikz}        % Creación de gráficos vectoriales.

% Configuración de enlaces y referencias cruzadas
\usepackage{hyperref}
\hypersetup{
    colorlinks   = true,
    linkcolor    = darkblue,
    citecolor    = black,
    filecolor    = blue,
    urlcolor     = blue
}

% Paquetes para la mejora visual de tablas y figuras
\usepackage{booktabs}    % Para tablas de alta calidad.
\usepackage{float}       % Controla la posición de figuras y tablas.

% Paquete para la personalización de códigos fuente
\usepackage{listings}
\lstset{
    literate=
    {á}{{\'a}}1 {é}{{\'e}}1 {í}{{\'i}}1 {ó}{{\'o}}1 {ú}{{\'u}}1
    {Á}{{\'A}}1 {É}{{\'E}}1 {Í}{{\'I}}1 {Ó}{{\'O}}1 {Ú}{{\'U}}1
    {ñ}{{\~n}}1 {Ñ}{{\~N}}1 {ü}{{\"u}}1 {Ü}{{\"U}}1,
    backgroundcolor=\color{backcolour},
    commentstyle=\color{codegreen},
    keywordstyle=\color{codepurple},
    numberstyle=\tiny\color{codegray},
    stringstyle=\color{red},
    basicstyle=\ttfamily\small,
    breakatwhitespace=false,
    breaklines=true,
    captionpos=b,
    keepspaces=true,
    numbers=left,
    numbersep=5pt,
    showspaces=false,
    showstringspaces=false,
    showtabs=false,
    tabsize=2,
    language=TeX,
    morecomment=[l]\#,
    frame=single,
    rulecolor=\color{black}
}

% Definición de colores al estilo Visual Studio Code
\definecolor{darkblue}{rgb}{0.0, 0.0, 0.55}  % Enlaces
\definecolor{codegreen}{rgb}{0.25, 0.49, 0.48}  % Comentarios
\definecolor{codegray}{rgb}{0.5, 0.5, 0.5}  % Números y anotaciones
\definecolor{codepurple}{rgb}{0.58, 0, 0.82}  % Palabras clave
\definecolor{backcolour}{rgb}{0.95, 0.95, 0.92}  % Fondo de código

% Configuraciones de párrafo y matemáticas
\usepackage{amsmath}
\usepackage{parskip}    % Espaciado entre párrafos.
\usepackage{ragged2e}   % Justificación mejorada.

% Configuración de secciones y encabezados
\usepackage{titlesec}
\titleclass{\part}{top} % Make part like a class
\titleformat{\part}[display]
  {\normalfont\huge\bfseries\centering}{\thepart}{20pt}{\Huge}
\titlespacing*{\part}{172.5pt}{-60pt}{10pt}
\titleformat{\part}
  {\normalfont\huge\bfseries}{}{0pt}{}

% Asegúrate de usar esto para mantener el estilo en las páginas de las partes
\titleformat{\part}[display]
  {\normalfont\huge\bfseries}{}{0pt}{}
  [\thispagestyle{fancy}] % Aplica el estilo fancy a las páginas de las partes

% Configuración avanzada de geometría
\geometry{
  paperwidth=21.6cm,  % Ancho del papel
  paperheight=27.9cm,  % Largo del papel
  left=3cm,  % Margen izquierdo
  right=2cm,  % Margen derecho
}

% Configuración de encabezados y pies de página personalizados
\usepackage{fancyhdr}
\pagestyle{fancy}
\fancyhf{}
%\fancyhead[L]{\raisebox{0.20cm}{\textbf{Métodos Computacionales en Obras Civiles}}}
%\fancyhead[R]{\raisebox{0.1cm}{\includegraphics[width=0.25\linewidth]{LOGO_UNIVERSIDAD.jpg}}}
%\fancyhead[C]{\rule{\textwidth}{0.6pt}}
%\fancyfoot[C]{\rule{\textwidth}{0.6pt}}
%\fancyfoot[R]{\raisebox{-1.5\baselineskip}{\thepage}}
\renewcommand{\headrulewidth}{0pt}
\renewcommand{\footrulewidth}{0pt}

\fancyhead[R]{\thepage} % Número de página a la derecha en el encabezado

% Configuracion de bibliografia
\usepackage{natbib}
\bibliographystyle{unsrtnat}  % Puedes cambiarlo por `unsrtnat`, `abbrvnat`, etc.

\begin{document}
%----------------------------------------------------------------------------------------
% PORTADA
%----------------------------------------------------------------------------------------
\begin{titlepage}%Inicio de la carátula, solo modificar los datos necesarios
\newcommand{\HRule}{\rule{\linewidth}{0.5mm}} 
\center 
%----------------------------------------------------------------------------------------
%	ENCABEZADO
%----------------------------------------------------------------------------------------
%----------------------------------------------------------------------------------------
%	SECCION DEL TITULO
%----------------------------------------------------------------------------------------
\begin{center}
  \textbf{\LARGE Titulo} \\[0.5cm]
  \textbf{Felipe Vicencio y Lukas Wolff} \\
  Facultad de Ingenieria y Ciencias Aplicadas, Universidad de los Andes, Santiago de Chile.\\
  email: \href{mailto:lwolff@miuandes.cl}{lwolff@miuandes.cl}, \href{mailto:favicencio@miuandes.cl}{favicencio@miuandes.cl}
  \\
  github: \href{https://github.com/LukasWolff2002/TAREA_3_AUTITOS}{Link al repositorio}
\end{center}

\vspace{1cm}

\begin{center}
  \textbf{\large ABSTRACT}    
\end{center}

\begin{justify}
  Hacer resumen aqui
  \\ \\ \\ \\ \\
  \textbf{Key Words}: 
\end{justify}


\vspace{1cm}

\end{titlepage}
%----------------------------------------------------------------------------------------
%	SECCION DEL AUTOR
%----------------------------------------------------------------------------------------

%----------------------------------------------------------------------------------------
%	SECCION DE LA FECHA
%----------------------------------------------------------------------------------------

%----------------------------------------------------------------------------------------
%  INDICE
%----------------------------------------------------------------------------------------

%----------------------------------------------------------------------------------------
%ACÁ EMPIEZA EL INFORME
\setcounter{page}{1}
%----------------------------------------------------------------------------------------

\section{Ressultados}

\subsection{Furness Matriz OD 2024}

Se aplico el metodo furness a la matriz OD del 2012, donde originalmente se tenian los siguientes datos:
{\footnotesize
  \[
  \begin{array}{cc}
  \text{$O_{2012}$} & \\
  \left[ \begin{array}{c}
  1960 \\
  6166 \\
  5150 \\
  25 \\
  2371 \\
  1 \\
  1387 \\
  1016 \\
  8150 \\
  8322
  \end{array} \right] &
  \left[ \begin{array}{cccccccccc}
  0 & 284 & 0 & 0 & 0 & 0 & 811 & 98 & 121 & 645 \\
  0 & 845 & 0 & 0 & 171 & 0 & 1622 & 836 & 1029 & 1663 \\
  0 & 0 & 107 & 0 & 0 & 0 & 0 & 0 & 1563 & 3480 \\
  0 & 0 & 0 & 0 & 0 & 0 & 0 & 25 & 0 & 0 \\
  0 & 1202 & 0 & 0 & 0 & 108 & 193 & 0 & 529 & 338 \\
  0 & 0 & 0 & 0 & 0 & 0 & 0 & 0 & 0 & 0 \\
  709 & 369 & 39 & 126 & 37 & 107 & 0 & 0 & 0 & 0 \\
  0 & 894 & 0 & 25 & 97 & 0 & 0 & 0 & 0 & 0 \\
  714 & 3514 & 1457 & 1208 & 728 & 529 & 0 & 0 & 0 & 0 \\
  821 & 3671 & 1886 & 949 & 344 & 650 & 0 & 0 & 0 & 0
  \end{array} \right]
  \end{array}
  \]
  \[
  \text{$D_{ 2012}$} \quad \left[ \begin{array}{cccccccccc}
    2245 & 10780 & 3488 & 2307 & 1377 & 1395 & 2626 & 959 & 3243 & 6126
  \end{array} \right]
  \]
}

Y se obtuvieron los siguientes resultados:

{\footnotesize
  \[
  \begin{array}{cc}
  \text{$O_{2024}$} & \\
  \left[ \begin{array}{c}
    8200 \\
    15991 \\
    7029 \\
    2337 \\
    4690 \\
    0 \\
    2611 \\
    953 \\
    3224 \\
    6090
  \end{array} \right] &
  \left[ \begin{array}{cccccccccc}
    0 & 346 & 0 & 0 & 0 & 0 & 1503 & 0 & 1685 & 4666 \\
    0 & 528 & 0 & 0 & 389 & 0 & 1543 & 0 & 7356 & 6175 \\
    0 & 0 & 67 & 0 & 0 & 0 & 0 & 0 & 3228 & 3734 \\
    0 & 0 & 0 & 0 & 0 & 0 & 0 & 2337 & 0 & 0 \\
    0 & 590 & 0 & 0 & 0 & 0 & 144 & 0 & 2970 & 986 \\
    0 & 0 & 0 & 0 & 0 & 0 & 0 & 0 & 0 & 0 \\
    1244 & 787 & 287 & 5 & 288 & 0 & 0 & 0 & 0 & 0 \\
    0 & 683 & 0 & 0 & 270 & 0 & 0 & 0 & 0 & 0 \\
    160 & 960 & 1373 & 6 & 725 & 0 & 0 & 0 & 0 & 0 \\
    339 & 1844 & 3268 & 9 & 630 & 0 & 0 & 0 & 0 & 0
  \end{array} \right]
  \end{array}
  \]
  \[
  \text{$D_{ 2024}$} \quad \left[ \begin{array}{cccccccccc}
    1743 & 5738 & 4995 & 20 & 2302 & 0 & 3190 & 2337 & 15239 & 15561
  \end{array} \right]
  \]
}

\textbf{Nota:} Se puede observar una ligera discrepancia en los datos, respecto a los vectores 2024 solicitados inicialmente, lo cual ocurre debido a que el metodo furnes no puede ser aplicado correctamente ya que el redondeo de datos de la matriz original ocaciona que algunons vectores no concuerdan al 100\%, vease el caso $O_{2012}$ = 1.


%\newpage
%\bibliography{referencias}  % Nombre del archivo .bib

\end{document}
